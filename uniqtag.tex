\documentclass{bioinfo}

% For \addlinespace and \bottomrule
\usepackage{booktabs}

% Place the caption above the table
\usepackage{float}
\floatstyle{plaintop}
\restylefloat{table}

\copyrightyear{2014}
\pubyear{2014}
\application % applications note

\begin{document}
\firstpage{1}

\title[UniqTag]{
UniqTag: Content-derived unique and stable identifiers for gene annotation}
\author[Jackman \textit{et al.}]{
Shaun Jackman$^{1,2,*}$, Joerg Bohlmann$^{3,4}$ and \.{I}nan\c{c} Birol$^{1,5}$
\footnote{to whom correspondence should be addressed}
}

\address{
$^1$Genome Sciences Centre, British Columbia Cancer Agency, Vancouver, BC, Canada
\\$^2$Graduate Program in Bioinformatics, University of British Columbia, Vancouver, BC, Canada
\\$^3$Michael Smith Laboratories, University of British Columbia, Vancouver, BC, Canada
\\$^4$Department of Forest Sciences, University of British Columbia, Vancouver, BC, Canada
\\$^5$Department of Medical Genetics, University of British Columbia, Vancouver, BC, Canada
}

\history{Received on XXXXX; revised on XXXXX; accepted on XXXXX}

\editor{Associate Editor: XXXXXXX}

\maketitle

\begin{abstract}
\section{Summary}\label{summary}

UniqTag assigns unique identifiers to gene sequences, or other arbitrary
sequences of characters, that are derived from the \emph{k}-mer
composition of the sequence. Unlike serial numbers, these identifiers
are stable between different assemblies and annotations of the same
data.

\section{Availability and
implementation}\label{availability-and-implementation}

The implementation of UniqTag is available at

\texttt{https://github.com/sjackman/uniqtag}

Supplementary data and code to reproduce it is available at

\texttt{https://github.com/sjackman/uniqtag-paper}

\section{Contact}\label{contact}

Shaun Jackman \textless{}sjackman@bcgsc.ca\textgreater{}

Inanc Birol \textless{}ibirol@bcgsc.ca\textgreater{}

\end{abstract}\section{Introduction}\label{introduction}

The task of annotating the genes of a genome sequence often follows
genome sequence assembly. These annotated genes are assigned unique
identifiers by which they can be referenced. Assembly and annotation is
often an iterative process, by refining the method or by the addition of
more sequencing data. These gene identifiers would ideally be stable
from one assembly and annotation to the next. Serial numbers are used
for identifiers of genes annotated by software such as MAKER
(\href{http://dx.doi.org/10.1104/pp.113.230144}{Campbell, 2014}), which,
although certainly unique, are not stable between assemblies. A single
change in the assembly can result in a total renumbering of the
annotated genes.

One solution to stabilize identifiers is to assign them based on the
content of the gene sequence. A cryptographic hash function such as SHA
(Secure Hash Algorithm)
(\href{http://www.nist.gov/manuscript-publication-search.cfm?pub_id=910977}{Dang,
2012}) derives a message digest from the sequence, such that two
sequences with the same content will have the same message digest, and
two sequences that differ will have different message digests. If a
cryptographic hash were used to identify a gene, the same gene in two
assemblies with identical content would be assigned identical
identifiers, but by design a slight change in the sequence, such as a
single-character substitution, would result in a completely different
digest and unique identifier.

A cryptographic hash function is designed so that small changes in the
message, even a single bit change, results in large changes to the
message digest: half of the bits of the digest are expected to flip,
called the avalanche effect
(\href{http://www.scientificamerican.com/article/cryptography-and-computer-privacy/}{Feistel,
1973}). Locality-sensitive hashing in contrast aims to assign items that
are similar to the same hash bucket. A hash function that, after a small
perturbation of the sequence, assigns an identical identifier to the
sequence is desirable for identifying the genes of a genome sequence
assembly project. One such locality-sensitive hash function, MinHash,
was employed in identifying web pages with similar content
(\href{http://dx.doi.org/10.1109/SEQUEN.1997.666900}{Broder, 1997}).
UniqTag implements MinHash, where the set of elements of an item is the
\emph{k}-mer composition of the sequence, the hash function is the
identity function and the minimal element is the lexicographically
minimal sequence, to assign stable identifiers to genes. These
identifiers are intended for systematic identification, unique within an
assembly, rather than as a biological name, which is typically assigned
based on biological function or homology to orthologous genes.

\section{Description}\label{description}

When iterating over multiple assemblies of the same data, it is rather
inconvenient when gene identifiers change from one assembly to the next.
UniqTag attempts to address this common challenge. By identifying the
gene by a feature of its content rather than an arbitrary serial number,
the gene identifier is stable between assemblies.

A UniqTag will change due to a difference in the locus of the UniqTag
itself, the creation of a least-frequent \emph{k}-mer that is
lexicographically smaller than the previous UniqTag, or the creation of
a \emph{k}-mer elsewhere resulting in the UniqTag no longer being a
least-frequent \emph{k}-mer. Concatenating two gene models results in a
gene whose UniqTag is the minimum of the two previous UniqTags, unless
one of the k-mer at the junction of the two sequences is
lexicographically smaller. Similarly when a gene model is split in two,
one gene is assigned a new UniqTag and the other retains the previous
UniqTag, unless the previous UniqTag spanned the junction.

A UniqTag can be generated from the nucleotide sequence of a gene. Using
instead the translated amino acid sequence of a protein-coding gene
sequence results in a UniqTag that is stable across synonymous changes
to the coding sequence as well as to changes in the untranslated regions
and introns of the gene. Since the amino acid alphabet is larger than
the nucleotide alphabet, fewer characters are required for a
\emph{k}-mer to be likely unique, resulting in an aesthetically pleasing
shorter identifier.

Two gene models with identical sequence would be assigned the same
UniqTag. It is possible that two genes that have no unique \emph{k}-mer
and similar \emph{k}-mer composition are assigned the same UniqTag.
Genes that have the same UniqTag are distinguished by adding a numerical
suffix to the UniqTag.

\subsection{Algorithm}\label{algorithm}

The UniqTag $u_k(s, S)$, a substring of the string \emph{s} with length
\emph{k} from the set of strings \emph{S}, is defined as follows.

$\Sigma$ is an alphabet. $\Sigma^k$ is the set of all strings over
$\Sigma$ of length \emph{k}. \emph{s} and \emph{t} are strings over
$\Sigma$. $C(s)$ is the set of all substrings of \emph{s}. A
\emph{k}-mer of \emph{s} is a substring of \emph{s} with length
\emph{k}. $C_k(s)$ is the set of all \emph{k}-mers of \emph{s}.

\[
C_k(s) = C(s) \cap \Sigma^k
\]

\emph{S} is a set of strings over $\Sigma$. $f(s, S)$ is the frequency
of \emph{s} in \emph{S}, defined as the number of strings in \emph{S}
that contain \emph{s} as a substring.

\[
f(s, S) = \left\vert \{ t \mid s \in C(t) \wedge t \in S \} \right\vert
\]

$\min S$ is the lexicographically minimal string of \emph{S}.
$u_k(s, S)$ is the UniqTag, the lexicographically minimal \emph{k}-mer
of those \emph{k}-mers of \emph{s} that are least frequent in \emph{S}.

\[
u_k(s, S) = \min \mathop{\arg\,\min}\limits_{t \in C_k(s)} f(t, S)
\]

\section{Results}\label{results}

UniqTag was used to assign seven-character identifiers to the protein
sequences of six builds of the Ensembl human genome
(\href{http://dx.doi.org/10.1093/nar/gkt1196}{Flicek, 2014}) spanning
five years. A seven-character string from the alphabet of the standard
20 amino acids allows for over a billion distinct identifiers. The
Ensembl human genome is used for its archive of historical annotations.
The overlap of UniqTag identifers between older builds and the current
build 75 is shown in Table~1. Comparing two recent builds 70 (Jan 2013)
and 75 (Feb 2014), fewer than 1\% of the UniqTags changed, and when
comparing a much older build 55 (Jul 2009) and the current build 75,
still over half the UniqTags are unchanged.

\begin{table}[!b]\centering\begin{tabular}[c]{@{}rrrrr@{}}
\toprule\addlinespace
Build A & Build B & Only in A & In both & Only in B
\\\addlinespace
\midrule
55 & 75 & 24299 & 30997 & 17600
\\\addlinespace
60 & 75 & 9365 & 34859 & 13738
\\\addlinespace
65 & 75 & 1088 & 45931 & 2666
\\\addlinespace
70 & 75 & 231 & 47955 & 642
\\\addlinespace
74 & 75 & 0 & 48597 & 0
\\\addlinespace
\bottomrule
\addlinespace
\caption{The overlap of UniqTag identifers between older builds of the
Ensembl human genome and the current build 75.}
\end{tabular}\end{table}

\section*{Acknowledgements}\label{acknowledgements}
\addcontentsline{toc}{section}{Acknowledgements}

The author thanks Nathaniel Street for his enthusiastic feedback, the
SMarTForests project and the organizers of the 2014 Conifer Genome
Summit that made our conversation possible.

\emph{Funding}: This work was supported by the Natural Sciences and
Engineering Research Council of Canada, Genome British Columbia, Genome
Alberta, Genome Quebec and Genome Canada.

\section*{References}\label{references}
\addcontentsline{toc}{section}{References}

\href{http://dx.doi.org/10.1109/SEQUEN.1997.666900}{Broder, A. Z.
(1997)} On the resemblance and containment of documents.
\emph{Compression and Complexity of Sequences}, 1997 Proceedings,
21-29.\\\href{http://dx.doi.org/10.1104/pp.113.230144}{Campbell, M. S.
\emph{et al.} (2014)} MAKER-P: a tool-kit for the rapid creation,
management, and quality control of plant genome annotations. \emph{Plant
Physiology}, 164(2),
513-524.\\\href{http://www.nist.gov/manuscript-publication-search.cfm?pub_id=910977}{Dang,
Q. H. (2012)} Secure Hash Standard (SHS). \emph{NIST FIPS}, 180(4),
1-35.\\\href{http://www.scientificamerican.com/article/cryptography-and-computer-privacy/}{Feistel,
H. (1973)} Cryptography and Computer Privacy. \emph{Scientific
American}, 228(5).\\\href{http://dx.doi.org/10.1093/nar/gkt1196}{Flicek,
P. \emph{et al.} (2014)} Ensembl 2014. \emph{Nucleic Acids Research},
42(D1), D749-D755.
\end{document}
